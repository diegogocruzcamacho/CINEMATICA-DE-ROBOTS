\documentclass[12pt,letterpaper]{article}
\usepackage[utf8]{inputenc}
\usepackage[spanish]{babel}
\usepackage{amsmath}
\usepackage{amsfonts}
\usepackage{amssymb}
\usepackage{graphicx}
\usepackage[left=4cm,right=3cm,top=4cm,bottom=3cm]{geometry}
\author{Diego Cruz Camacho}
\title{Tarea1}
\begin{document}
\maketitle
\begin{center}
\includegraphics[scale=1.2]{logo.jpg}\\

Nombre: Cruz Camacho Diego

Materia: Cinematica de Robtos

Carrera: Ing. Mecatrónica

Docente: Ing. Carlos Enrique Moran Garabito

Grado y Grupo: 7mo B\\           

\end{center}

\begin{center}
•\textbf{INVESTIGACIÓN DE PAR DE ROTACIÓN Y CUATERNIOS}\\
\end{center}

\begin{center}
\textit{Par de Rotación}\\
\end{center}

\begin{flushleft}

Una matriz de rotación 3x3 se puede definir como una matriz de transformación que operando, en un espacio Euclídeo tridimencional, sobre un sistema de coordenadas móvil centrado en el origen, origina una rotación del mismo. Se puede definir también como la relación entre dos sistemas de coordenadas con orígenes coincidentes. En este segundo caso, la aplicación de la matriz de rotación permitirá conocer la relación entre los componentes de un vector dado con respecto a ambos sistemas de coordenadas.
\end{flushleft}
\begin{center}

\includegraphics[scale=1]{build/matriz.jpg}\\  


\textit{Par de Cuaternios}\\
\end{center}

\begin{flushleft}



Los cuaternios unitarios proporcionan una notación matemática para representar las orientaciones y las rotaciones de objetos en tres dimensiones. Comparados con los ángulos de Euler, son más simples de componer y evitan el problema de bloque del cardán. Comparados con las matrices de rotación, son más eficientes y más estables numéricamente. Los cuarteniones son útiles en aplicaciones de gráficos por computadora, robótica, navegación y mecánica orbital de satélites.
\begin{flushleft}


\includegraphics[scale=0.8]{eulerangles.jpg} 

En robótica es necesario guardar las orientaciones de los sistemas. Sin embargo se sabe que con tres ángulos se puede definir una orientación. De este modo se calcula la orientación como una matriz de rotación pero no se almacena ésta, sino los tres ángulos que la definen y con los cuales se puede definir la matriz de rotación. Normalmente, el modo elegido para representar la orientación del factor final del robot con respecto a un sistema de coordenadas de referencia emplea los llamados ángulos de Euler alpha,beta,gamma. Aunque los ángulos de Euler describen la orientación de un cuerpo rígido con respecto a un sistema de coordenadas fijo, hay diferentes tipos de representación de ángulos de Euler.

\end{flushleft}
\end{flushleft}



\end{document}